\documentclass[a4paper]{scrartcl}

\usepackage[ngerman]{babel}
\usepackage[utf8]{inputenc}
\usepackage[scale=0.8]{geometry}
\usepackage{amsmath}

\author{Benedikt Sauer}
\title{N-Körperproblem}

\begin{document}
\maketitle
\begin{abstract}
  Dieser Vortrag behandelt die Implementierung und Verwendung von
  N-Körpersimulationen zur Berechnung der Verteilung dunkler Materie im
  Universum.
\end{abstract}
\section{Einführung}
Das Mehrkörperproblem besteht allgemein aus mehreren paarweise untereinander
wechselwirkenden Körper. Für den Fall von genau zwei Körpern (Keplerproblem) ist
es vollständig (und elementar) lösbar. Für mehr als zwei Körpern existiert zwar
eine analytische Lösung (\cite{wang}), diese ist allerdings mehr von
theoretischer Bedeutung. Hier wird die numerische Lösung betrachtet.

Die Hamiltonfunktion des Problems ist
\begin{align}
  \mathcal{H}(p_i, x_i) = \sum_i \frac{\left|p_i\right|}{2m_i} +
  \frac{1}{2} \sum_{i,j} m_i m_j \varphi(x_i - x_j)
  \label{eq:hamilton}
\end{align}
Das darin benötigte Potential erfüllt die Poissongleichung:
\begin{align}
  \Delta \varphi(x) = 4\pi G \rho(x)\text{, mit }\rho(x) = \sum_{i=1}^n m_i
  \delta_d(x - x_i)
  \label{eq:poisson}
\end{align}

Die Lösung selbiger ist das Hauptproblem der Simulationen. Hierzu gibt es
verschiedene Herangehensweisen:
\begin{description}
  \item[PP-Methode (Particle-Particle)]
    Hierbei werden die Kräfte auf ein Teilchen explizit ausgerechnet, indem
    sämtliche Beiträge aller anderen aufaddiert werden. die Komplexität ist also
    quadratisch ($\mathcal O(n^2)$), was ein er<+label+>heblicher Nachteil ist.
  \item[PM-Methode (Particle-Mesh)]
    Der Raum wird hierbei mit einem Gitter gefüllt. Für jeden einzelnen
    Gitterpunkt wird dann eine Dichte ausgerechnet. Anschließend wird dieses
    Gitter fouriertransformiert, wodurch sich die partielle
    Differentialgleichung zu einer algebraischen Gleichung vereinfacht. Diese
    wird dann im transformierten Raum gelöst und wieder per FFT
    zurücktransformiert. Die Methode hat ein deutlich besseres Laufzeitverhalten
    als PP ($\mathcal O(n\log n)$), unterschätzt aber auf geringe Distanzen
    zwangsläufig durch die geringe Auflösung die Kräfte.

\end{description}

\section{Implementierung in Gadget}


\section{Zusammenfassung}

%\begin{thebibliography}
%    \begin{biblist}
%      bla
%    \end{biblist}
%\end{thebibliography}

\end{document}
