\documentclass{beamer}

\mode<presentation>
{
  \usetheme{Warsaw}
  \setbeamercovered{transparent}
}


\usepackage[ngerman]{babel}
\usepackage[utf8]{inputenc}
%\usepackage{times}
%\usepackage[T1]{fontenc}

\title[Kurzversion des Titels]{Titel}

\subtitle{Untertitel}

\author{Benedikt Sauer}

\institute{}

\date[Proseminar Präsentation]{Proseminar Präsentation - 22.01.2009}

\subject{Physik}

% Falls eine Logodatei namens "university-logo-filename.xxx" vorhanden
% ist, wobei xxx ein von latex bzw. pdflatex lesbares Graphikformat
% ist, so kann man wie folgt ein Logo einfuegen:

% \pgfdeclareimage[height=0.5cm]{university-logo}{university-logo-filename}
% \logo{\pgfuseimage{university-logo}}

% Folgendes sollte geloecht werden, wenn man nicht am Anfang jedes
% Unterabschnitts die Gliederung nochmal sehen moechte.
\AtBeginSubsection[]
{
  \begin{frame}<beamer>{Gliederung}
    \tableofcontents[currentsection,currentsubsection]
  \end{frame}
}


% Falls Aufzaehlungen immer schrittweise gezeigt werden sollen, kann
% folgendes Kommando benutzt werden:

\beamerdefaultoverlayspecification{<+->}

\begin{document}

\begin{frame}
  \titlepage
\end{frame}

\begin{frame}{Gliederung}
  \tableofcontents
  % Die Option [pausesections] koennte nuetzlich sein.
\end{frame}

% - Es sollte genau zwei oder drei Abschnitte geben (neben der
%   Zusammenfassung). 
% - *Hoechstens* drei Unterabschnitte pro Abschnitt.
% - Pro Rahmen sollte man zwischen 30s und 2min reden. Es sollte also
%   15 bis 30 Rahmen geben.

\section{Einführung}
\begin{frame}{}{Untertitel sind optional.}
  % - Eine Überschrift fasst einen Rahmen verständlich zusammen. Man
  %   muss sie verstehen können, selbst wenn man nicht den Rest des
  %   Rahmens versteht.

  \begin{itemize}
  \item
    Viel \texttt{itemize} benutzen.
  \item
    Sehr kurze Sätze oder Satzglieder verwenden.
  \end{itemize}
\end{frame}

\begin{frame}{Überschriften müssen informativ sein.}

  Man kann Overlays erzeugen\dots
  \begin{itemize}
  \item mit dem \texttt{pause}-Befehl:
    \begin{itemize}
    \item
      Erster Punkt.
      \pause
    \item    
      Zweiter Punkt.
    \end{itemize}
  \item
    mittels Overlay-Spezifikationen:
    \begin{itemize}
    \item<3->
      Erster Punkt.
    \item<4->
      Zweiter Punkt.
    \end{itemize}
  \item
    mit dem allgemeinen \texttt{uncover}-Befehl:
    \begin{itemize}
      \uncover<5->{\item
        Erster Punkt.}
      \uncover<6->{\item
        Zweiter Punkt.}
    \end{itemize}
  \end{itemize}
\end{frame}


\subsection{Physik}
\subsubsection{Gravitation}
\subsubsection{Hydrodynamik}
\subsection{Mathematik}
\subsubsection{Poissongleichung}
\subsubsection{Lösbarkeit}
\subsection{Informatik}
\subsubsection{Parallelisierung}

\section{Implementierung (Gadget)}
\subsection{Diskretisierung}
\subsection{Charakteristikenverfahren}
\subsection{Particle Mesh}
\subsubsection{allgemein}
\subsubsection{P3M}
\subsubsection{TreePM}

\section{andere Verfahren}
\subsection{bla}

\section{Zusammenfassung}
\begin{frame}{Zusammenfassung}

  % Die Zusammenfassung sollte sehr kurz sein.
  \begin{itemize}
  \item
    Die \alert{erste Hauptbotschaft} des Vortrags in ein bis zwei Zeilen.
  \item
    Die \alert{zweite Hauptbotschaft} des Vortrags in ein bis zwei Zeilen.
  \item
    Eventuell noch eine \alert{dritte Botschaft}, aber nicht noch mehr.
  \end{itemize}
  
  % Der folgende Ausblick ist optional.
  \vskip0pt plus.5fill
  \begin{itemize}
  \item
    Ausblick
    \begin{itemize}
    \item Parallelisierung (OpenCV?!)
    \item
      Nochwas, das wir noch nicht lösen konnten.
    \end{itemize}
  \end{itemize}
\end{frame}

\end{document}


