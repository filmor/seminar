\documentclass{beamer}

\mode<presentation>
{
  \usetheme{Berlin}
  \usefonttheme{professionalfonts}
  \setbeamercovered{transparent}
  \setbeamertemplate{footline}[frame number]
}

\usepackage[ngerman]{babel}
\usepackage[utf8]{inputenc}
\usepackage{times}
\usepackage[T1]{fontenc}

\title[N-Körperproblem I]{N-Körperproblem I}
% \subtitle{}

\author[B. Sauer]{Benedikt Sauer}

\institute{}

\date[Proseminar Präsentation]{Proseminar Präsentation - 22.01.2009}

\subject{Physik}

% Falls eine Logodatei namens "university-logo-filename.xxx" vorhanden
% ist, wobei xxx ein von latex bzw. pdflatex lesbares Graphikformat
% ist, so kann man wie folgt ein Logo einfuegen:

% \pgfdeclareimage[height=0.5cm]{university-logo}{university-logo-filename}
% \logo{\pgfuseimage{university-logo}}

% Folgendes sollte geloecht werden, wenn man nicht am Anfang jedes
% Unterabschnitts die Gliederung nochmal sehen moechte.
%\AtBeginSubsection[]
%{
%  \begin{frame}<beamer>{Gliederung}
%    \tableofcontents[currentsection,currentsubsection]
%  \end{frame}
%}


% Falls Aufzaehlungen immer schrittweise gezeigt werden sollen, kann
% folgendes Kommando benutzt werden:

\beamerdefaultoverlayspecification{<+->}

\begin{document}

\begin{frame}
  \titlepage
\end{frame}

\begin{frame}{Gliederung}
  \tableofcontents[pausesections]
  % Die Option [pausesections] koennte nuetzlich sein.
\end{frame}

% - Pro Rahmen sollte man zwischen 30s und 2min reden. Es sollte also
%   15 bis 30 Rahmen geben.

\section{Einführung}
\begin{frame}[Das Mehrkörperproblem]
  \begin{itemize}
    \item Mehrere untereinander wechselwirkende Körper
    \item Interessant (weil nicht mehr einfach lösbar) für mehr als 2 Körper
      \begin{itemize}
        \item \alert{nicht} unlösbar, es existiert eine analytische Lösung
          (vorausgesetzt, die Körper kollidieren nicht)
      \end{itemize}
    \item Für große Zahlen müssen wir numerisch lösen
  \end{itemize}
  \vskip10pt
  \begin{itemize}
    \item Und genau das wollen wir nun tun\ldots
  \end{itemize}
\end{frame}
\subsection{Physik}
\subsubsection*{Gravitation}
\begin{frame}{Gravitation}
  Im einfachsten Fall haben wir folgende Hamiltonfunktion:
  \begin{align}
    \mathcal H(p_i,x_i) = \sum_i \frac{\vec{p}_i^2}{2m_i} +
    \frac{1}{2}\sum_{i,j} {m_i m_j \varphi(\vec{x}_i - \vec{x}_j)}
    \label{eqn:hamilton}
  \end{align}
  wobei $\varphi$ das Gravitationspotential ist.
\end{frame}
\begin{frame}{relativistische Ergänzungen}
  Da wir aufgeklärte, relativistisch rechnenden Menschen sind, wird die
  Hamiltonfunktion durch den Skalierungsfaktor $a(t)$ explizit zeitabhängig:
  \begin{align}
    \mathcal H(p_i,x_i,t) = \sum_i \frac{\vec{p}_i^2}{2m_i a(t)^2} +
    \frac{1}{2}\sum_{i,j} \frac{m_i m_j \varphi(\vec{x}_i - \vec{x}_j)}{a(t)}
    \label{eqn:hamilton_rel}
  \end{align}
  mit den neuen kanonischen Impulsen $\vec p_i := a(t)^2 m_i\dot\vec{x}_i$.
  Wir skalieren die Wirkung unserer Abstände auf die Energie um.
\end{frame}
\subsection{Mathematik}
\begin{frame}{Poissongleichung}
  Das Potential $\varphi$ erhalten wir aus der \emph{Poissongleichung}
  \begin{align}
    \Delta \varphi(\vec{x}) = 4\pi G \rho(\vec{x})
  \end{align}
  mit einer Dichtefunktion $\rho$, die im einfachsten Fall von $n$ Körpern
  \begin{align}
    \rho(x) = \sum_{i=1}^n m_i \delta(\vec{x} - \vec{x}_i)
  \end{align}
  ist.
\end{frame}
\begin{frame}{Poissongleichung}{Diskretisierung}
\end{frame}
\begin{frame}{Poissongleichung}{Hydrodynamik}
\end{frame}
% Als nächstes: Diskretisierung der Poissongleichung
% SPH-Methode

\begin{frame}
  
\end{frame}

\subsection{Informatik}
\subsubsection*{Parallelisierung}
\begin{frame}{Parallelisierung}
  Teile-und-Herrsche-Prinzip
  % Ausführen
\end{frame}
\begin{frame}
  \begin{itemize}
    \item Parallelisierung ist toll:
      \begin{itemize}
        \item Wir brauchen nur vergleichsweise kleine Rechner
        \item Wir ersparen uns Flaschenhälse
      \end{itemize}
    \item Also wollen wir unsere Algorithmen parallel ausführen
  \end{itemize}
\end{frame}

% Octtree? Baumstruktur?

\section{Implementierung (Gadget)}
\subsection{Gravitationsberechnung}
\subsubsection*{Diskretisierung}
\subsubsection*{Charakteristikenverfahren}
\subsubsection{Particle-Particle}
\subsubsection{Particle-Mesh}
\subsubsection*{allgemein}
\subsubsection*{P3M}
\subsubsection*{TreePM}
\subsection{Zeitentwicklung}
\subsection{Parallelisierung}

\section{Zusammenfassung}
\begin{frame}{Zusammenfassung}

  % Die Zusammenfassung sollte sehr kurz sein.
  \begin{itemize}
  \item
    Wir können mittlerweile mit sehr hoher Genauigkeit und gleichzeitig hoher
    Geschwindigkeit rechnen
  \item
    Guckstu nächster Vortrag
  \end{itemize}
  
  % Der folgende Ausblick ist optional.
  \vskip0pt plus.5fill
  \begin{itemize}
  \item
    Ausblick
    \begin{itemize}
    \item
      weitere Parallelisierung (z.B. OpenCL)\\
      $\Rightarrow$ billigere Rechner und insgesamt mehr Leistung
    \item

    \end{itemize}
  \end{itemize}
\end{frame}

\end{document}


