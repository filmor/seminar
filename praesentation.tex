\documentclass{beamer}

\mode<presentation>
{
  \usetheme{Singapore}
  \usefonttheme{professionalfonts}
  \setbeamercovered{transparent}
  \setbeamertemplate{footline}[frame number]
}

\usepackage[ngerman]{babel}
\usepackage[utf8]{inputenc}
\usepackage{multimedia}

\title[N-Körperproblem I]{N-Körperproblem I}
% \subtitle{}

\author[B. Sauer]{Benedikt Sauer}

\institute{}

\date[Proseminar Präsentation]{Proseminar Präsentation - 22.01.2009}

\subject{Physik}

% Falls eine Logodatei namens "university-logo-filename.xxx" vorhanden
% ist, wobei xxx ein von latex bzw. pdflatex lesbares Graphikformat
% ist, so kann man wie folgt ein Logo einfuegen:

% \pgfdeclareimage[height=0.5cm]{university-logo}{university-logo-filename}
% \logo{\pgfuseimage{university-logo}}

% \beamerdefaultoverlayspecification{<+->}

\begin{document}

\begin{frame}
  \titlepage
\end{frame}

\begin{frame}{Gliederung}
  \tableofcontents[pausesections]
\end{frame}

\section{Einführung}
\subsection{Physik}
\begin{frame}{Motivation}
  \begin{itemize}
    \item Materieverteilung
    \item $\Rightarrow$ kosmische Parameter ($\Lambda$, etc.)
      \pause
      \vskip10pt
    \begin{block}{Boltzmanngleichung}
      \begin{align}
        \frac{\mathrm df}{\mathrm dt} = \frac{\partial f}{\partial t} + \vec v
        \frac{\partial f}{\partial r} - \frac{\partial \varphi}{\partial \vec
        r}\frac{\partial f}{\partial \vec v} = 0
      \end{align}
    \end{block}
      \pause
    \item Phasenraumvolumen konstant
      \pause
    \item verwenden Monte-Carlo-Ansatz mit Tracerkörpern
  \end{itemize}
\end{frame}

\begin{frame}{Das Mehrkörperproblem}
  \begin{itemize}
    \item Mehrere untereinander wechselwirkende Körper
      \pause
    \item Interessant (weil nicht mehr einfach lösbar) für mehr als 2 Körper
      \begin{itemize}
        \item \alert{nicht} unlösbar, es existiert eine analytische Lösung
          (vorausgesetzt, die Körper kollidieren nicht) [Wang, 1990]
      \end{itemize}
      \pause
    \item Für große Zahlen müssen wir numerisch lösen
  \end{itemize}
  \vskip10pt
  \begin{itemize}
    \item Und genau das wollen wir nun tun\ldots
  \end{itemize}
\end{frame}

\begin{frame}{Hamiltonfunktion}
  Im einfachsten Fall haben wir folgende Hamiltonfunktion:
  \begin{align}
    \mathcal H(p_i,x_i) = \sum_i \frac{\left|\vec{p}_i\right|^2}{2m_i} +
    \frac{1}{2}\sum_{i,j} {m_i m_j \varphi(\vec{x}_i - \vec{x}_j)}
    \label{eqn:hamilton}
  \end{align}
  wobei $\varphi$ das Gravitationspotential ist.
\end{frame}

\begin{frame}{relativistische Ergänzungen}
  Für die Berechnungen verwenden wir \alert{mitbewegende Koordinaten}
  \begin{align}
    \mathcal H(p_i,x_i,t) = \sum_i \frac{\vec{p}_i^2}{2m_i a(t)^2} +
    \frac{1}{2}\sum_{i,j} \frac{m_i m_j \varphi(\vec{x}_i - \vec{x}_j)}{a(t)}
    \label{eqn:hamilton_rel}
  \end{align}
  mit den neuen kanonischen Impulsen $\vec p_i := a(t)^2 m_i\dot{\vec{x}}_i$.
  \pause
  \vskip10pt
  Wir skalieren die Wirkung unserer Abstände auf die Energie um.
\end{frame}

\subsection{Mathematik und Informatik}
\begin{frame}{Poissongleichung}
  Das Potential $\varphi$ erhalten wir aus der \emph{Poissongleichung}
  \begin{align}
    \Delta \varphi(\vec{x}) = 4\pi G \rho(\vec{x})
  \end{align}
  mit einer Dichtefunktion $\rho$\pause, im einfachsten Fall von $n$ Körpern
  \begin{align}
    \rho(\vec x) = \sum_{i=1}^n m_i \delta_d(\vec{x} - \vec{x}_i)
  \end{align}
\end{frame}

\begin{frame}{Poissongleichung}{Diskretisierung}
  \begin{itemize}
    \item<1-> Wir nehmen uns den Differenzenquotienten und streichen den Limes weg:
      \begin{align}
        f'(x) = \lim_{h\to 0} \frac{f(x+h) - f(x)}{h} \quad \to f_h'(x) =
        \frac{f(x + h) - f(x)}{h}
      \end{align}
    \item<2-> Das können wir auch für die zweite Ableitung machen
    \item<3-> Erhalten daraus mit $\nabla \cdot \nabla f = \Delta f$ eine Matrix für
      den Laplaceoperator (Bandmatrix)
  \end{itemize}
\end{frame}
% Als nächstes: Diskretisierung der Poissongleichung

\begin{frame}{Baumalgorithmen und Parallelisierung}
  \begin{itemize}
    \item<1-> Wir benötigen für das Programm eine passende Raumaufteilung
      \begin{itemize}
        \item<2-> Benutzen einen \alert{Octree} \pause
        \item<3-> Teilen den Raum in Würfel auf, bis jeder Würfel genau ein Teilchen
          enthält \pause
        \item<4-> Leere Würfel werden nicht gespeichert \pause
      \end{itemize}
      \vskip20pt
    \item<5-> Damit wir das Programm in angemessener Zeit laufen lassen können,
      führen wir es parallel aus
      \begin{itemize}
        \item Bäume lassen sich gut parallelisieren
        \item<6-> Die FFT, die wir ebenfalls benötigen
      \end{itemize}
  \end{itemize}
\end{frame}

\section{Implementierung (Gadget)}
\subsection{Gravitationsberechnung}
\begin{frame}{PP-Methode (Particle-Particle)}
  \begin{itemize}
    \item Wir berechnen die Kraft auf ein Teilchen explizit (führen die
      Doppelsumme aus) \pause \\
      \begin{block}{zur Erinnerung}
        \begin{align}\sum_{i,j} m_i m_j \varphi(\vec x_i - \vec x_j)\end{align}
      \end{block}
    \item Grauslige Laufzeit: $\mathcal{O}(n^2)$
      \begin{itemize}
        \item Verdoppeln der Teilchenanzahl $\Rightarrow$ Vervierfachung des
          Aufwandes \pause
        \item Nur für relativ kleine Zahlen tatsächlich machbar
      \end{itemize}
  \end{itemize}
\end{frame}

\begin{frame}{Baumalgorithmus}
  Wir führen das Folgende für jedes Teilchen aus:\vskip10pt
  \begin{block}{Algorithmus}
    \begin{itemize}
      \item Laufe den Baum herab \pause \vskip10pt
      \item Sind wir weit genug weg oder bei einem Blatt? \pause
        \begin{itemize}
          \item Falls ja: Berechne die Masse des Teilbaumes
          \item Ansonsten führe den Algorithmus für alle Kinder aus
        \end{itemize}
    \end{itemize}
  \end{block}
\end{frame}

\begin{frame}{Baumalgorithmus}
  \begin{itemize}
    \item Die Genauigkeit hängt von unserer Abbruchbedingung ab \pause
    \item Die Geschwindigkeit ebenfalls \pause
    \item Im schlimmsten Fall sind wir wegen Overhead sogar langsamer als PP
      und wir benötigen mehr Speicher
      \pause
      \vskip20pt
    \item Bei anständigen Abbruchbedingungen: Laufzeit $\mathcal{O}(n\log n)$
  \end{itemize}
\end{frame}

\begin{frame}{PM-Methode (Particle Mesh)}
  \begin{itemize}
    \item Legen ein Gitter über unseren Raum
      \begin{itemize}
        \item Die Knoten enthalten die "`Dichte"'
      \end{itemize}
      \pause
      \vskip10pt
    \item Fouriertransformation (in Gadget: FFTW)
      \begin{itemize}
        \item Wie zuvor, $\Delta \varphi(\vec x) \to
          -\left|\vec k\right|^2\hat{\varphi}(\vec k)$
        \item Lösen die algebraische Gleichung
          \vskip5pt
        \item Transformieren zurück mit dem selben Algorithmus
      \end{itemize}
      \pause
      \vskip10pt
    \item Auf kurze Reichweiten ungenau aufgrund der Diskretisierung
  \end{itemize}
\end{frame}

\subsection{Hybride Verfahren}
\begin{frame}{$\text{P}^3\text{M}$-Methode (Particle-Particle + Particle Mesh)}
  \begin{itemize}
    \item<1-> Können aufgrund der Linearität der Poissongleichung aufteilen in 
      $\varphi = \varphi_{\text{kurz}} + \varphi_{\text{weit}}$
      \vskip10pt
      \begin{itemize}
        \item<2-> Beim kurzreichweitigen Teil tragen nur vergleichsweise wenige
          Teilchen bei $\Rightarrow$ PP-Methode 
        \item<2-> Die Fernwirkung ist durch die PM-Methode hinreichend genau
          berechenbar
      \end{itemize}
      \vskip20pt
    \item<3-> Leider immernoch zu langsam aufgrund von Halo-Bildung \pause
    \item<4-> Kaum besser als PP
  \end{itemize}
\end{frame}

\begin{frame}{TreePM-Methode (Baum + Particle Mesh)}
  \begin{itemize}
    \item Selber Ansatz wie zuvor, benutzen aber für $\varphi_{\text{kurz}}$
      unseren Baumalgorithmus
      \vskip20pt
    \item $\Rightarrow$ im besten Fall $\mathcal{O}(n\log n)$-Laufzeit im
      nahreichweitigen Fall \pause
    \item sehr gute Ergebnisse und deutlich bessere Laufzeit
  \end{itemize}
\end{frame}

\subsection*{Zeitentwicklung}
\begin{frame}{Zeitentwicklung}
  \begin{itemize}
    \item<+-> Wir haben nun unsere Kräfte berechnet und wollen das System entwickeln
    \item<+-> Müssen wiedermal eine Differentialgleichung numerisch lösen, diesmal
      die Bewegungsgleichung
    \item<+-> Ist eine gewöhnliche Differentialgleichung $\Rightarrow$ zum Beispiel
      Leapfrogverfahren
      \begin{itemize}
        \item<+-> Berechnen abwechselnd die Position und die Geschwindigkeit
        \item<+-> $x_{i+1} = x_i + v_{i}\Delta t + a_i\frac{\Delta t^2}{2}$
        \item<+-> $v_{i+1} = v_{i+1} + \frac{a_i + a_{i+1}}{2}\Delta t$
      \end{itemize}
    \item<+-> Genauigkeit lässt sich auch noch steigern (Runge-Kutta, adaptive
      Verfahren)
  \end{itemize}
\end{frame}

\section{Ergebnisse}
\subsection*{Millennium-Simulation}
\begin{frame}{Millennium-Simulation}
  Berechnung der Verteilung der dunklen Materie im Universum
  \begin{itemize}
    \item Würfelförmiges Gebiet von $2\cdot 10^9$ Lichtjahren Kantenlänge
    \item Start der Simulation ist bei der Aussendung der Hintergrundstrahlung
      mit entsprechenden Anfangsbedingungen (Massenverteilung)
    \item 10 Milliarden berechnete Tracerteile von jeweils etwa 100 Millionen
      Sonnenmassen
      \pause
    \item Daraus ergaben sich etwa 20 Millionen "`Galaxien"'\pause
  \end{itemize}
\end{frame}

\subsection*{Aquarius-Projekt}
\begin{frame}{Aquarius}
  Nehmen einen der Halos aus einer vorigen Berechnung und rechnen diesen in
  höherer Auflösung weiter
  \vskip10pt
  \begin{itemize}
    \item Größe in etwa so wie die der Milchstraße
      \vskip10pt
    \item Zeigt sehr viele Klumpen um die Galaxie herum
  \end{itemize}
\end{frame}

\section{Zusammenfassung}
\begin{frame}{Zusammenfassung}
  \begin{itemize}
  \item Wir können mittlerweile mit sehr hoher Genauigkeit und gleichzeitig
    hoher Geschwindigkeit rechnen
  \item Erhalten damit wichtige Hinweise, ob die aktuellen Theorien stimmen
    können
  \item Für weiteres: Siehe nächster Vortrag
  \end{itemize}
\end{frame}
\end{document}

